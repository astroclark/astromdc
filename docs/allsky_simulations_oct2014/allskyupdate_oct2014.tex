\documentclass{beamer}
\setbeamertemplate{navigation symbols}{}

\usepackage{beamerthemeshadow}
\setbeamertemplate{caption}[numbered]

\hypersetup{colorlinks}

\def\gw#1{gravitational wave#1 (GW#1)\gdef\gw{GW}}
\def\ns#1{neutron star#1 (NS#1)\gdef\ns{NS}}

\newcommand{\red}[1]{{\color{red}{#1}}}

\begin{document}
\title{(Some) Waveform Simulation [MDC] Infrastructure}
\subtitle{All-Sky Burst Call Oct 21$^{\text{st}}$ 2014}  
\author{James A. Clark}
\institute{Georgia Institute Of Technology}
\date{} 

\begin{frame}[plain]
\titlepage
\end{frame}

%\begin{frame}\frametitle{Table of contents}\tableofcontents
%\end{frame} 

\section{{\tt LALSimulation} \& Burst Injections}

\begin{frame}
    \frametitle{Bursts In LALSimulation}
    {\tt LALSimulation}: simulation engine for well-defined, analytic waveforms.
    Can make the following \emph{right now}:
    \begin{itemize}
        \item sine-Gaussians
        \item white noise bursts
        \item Gaussians (?)
        \item Cosmic strings
        \item \dots
    \end{itemize}
    Astrophysical waveforms will require a little more work:
    \begin{itemize}
        \item Supernovae
        \item Binary coalescence with matter
        \item NR BBH (!)
    \end{itemize}
\end{frame}

\begin{frame}
    \frametitle{MDC Generation With LALSimulation}
    Utilities in excess-power for e.g.,:
    \begin{itemize}
        \item MDC frames with user-specified distributions on optimal-SNR
            $h_{\text{rss}}$, frequencies, quality factors, bandwidths,
            durations etc
        \item Distributions created by {\tt lalapps\_binj} - creates XML file
            with a {\tt sim\_burst} which is readable by {\tt LAL} codes.
        \item Already deployed in ETG comparison study, cosmic string search 
    \end{itemize}
    Useful links:
    \begin{itemize}
        \item ETG trigger comparison wiki:
            \href{https://wiki.ligo.org/viewauth/DetChar/ETGperformanceStudy}{https://wiki.ligo.org/viewauth/DetChar/ETGperformanceStudy}
        \item Excess-power utilities:
            \href{https://github.com/cpankow/excesspower-utils}{https://github.com/cpankow/excesspower-utils}
    \end{itemize}
\end{frame}

\begin{frame}
    \frametitle{Astrophysical Waveforms: NINJA Codes}
    {\tt LALSimulation} and friends: parameterised waveforms.  No current
    support for waveforms directly from numerical simulation (SNe, NR BBH, NSBH,
    BNS).
    \begin{itemize}
        \item Obvious candidate: {\tt NINJA} codes\footnote{Currently broken,
            unmaintained\dots}
        \item NINJA infrastructure: tools and standards for handling NR BBH data in a
            standard format, project signals with random extrinsic params onto
            detectors.
        \item Currently only support binary mergers (i.e., {\tt sim\_inspiral} tables);
            would be convenient to add support to {\tt sim\_burst} for e.g., SNe
            \& injection-finding / characterisation using burst codes
        \item {\tt NINJA} codes were used in IMR study, S6 IMBH analysis,
            post-BNS merger studies (and NINJA!)
    \end{itemize}
\end{frame}

\begin{frame}
    \frametitle{LALSimulation \& Swig}
    Worth noting injections easily performed in python using swig-wrapped LAL
    routines:
    \begin{enumerate}
        \item Construct LAL TimeSeries objects for $h_+$, $h_{\times}$ (e.g.,
            read from file or generated with {\tt LALSimulation})
        \item Generate sky-location, polarisation, detector site (can e.g., be
            read from {\tt sim\_burst, sim\_inspiral} tables)
        \item {\tt SimDetectorStrainREAL8TimeSeries()} projects $h_{+,\times}$
            onto this detector with these angles
        \item Can then pass the TimeSeries or python arrays directly on for
            further analysis or write to frame with e.g. pylal frame library
    \end{enumerate}
\end{frame}

\begin{frame}
    \frametitle{LALSim \& Swig}
    Currently using LALSim/Swig in LIB post-merger studies:
    \begin{enumerate}
        \item Read quadrupole moments $\ddot{I_{xx}}, \ddot{I_{xy}}, \dots$ from
            file
        \item Construct expansion parameters $H_{lm}$ (see LIGO-T1000553)
        \item Generate random $(\theta,~\phi)$ and construct
            \begin{equation}
                h_+ - ih_{\times} = \frac{1}{D} \sum_{l=2}^{\infty}
                \sum_{m=-2}^{m=2} H_{lm}(t) ^{-2}Y_{lm}(\theta, \phi)
            \end{equation}
        \item Choose detector, sky-location \& project $h_{+,\times}$ onto
            detector
        \item Write to frame, generate cache file
        \item Call {\tt lalinference\_nest} with {\tt subprocess.call()}
        \item Delete frame, cache
        \item repeat (3--7)\dots
    \end{enumerate}
\end{frame}

\begin{frame}
    \frametitle{Summary}
    \begin{itemize}
        \item {\tt LALSimulation} + {\tt LALBurst} + {\tt GSTLAL} =  well
            documented, largely reviewed and easy way to generate MDCs
        \item Can be run by \emph{anyone} with LDG access (i.e., software is
            installed system-wide)
        \item Used in ETG trigger study \& cosmic string analysis
        \item Only standard ad hoc burst waveforms (+strings) currently supported
        \item NINJA can be used for unparameterized, astrophysical waveforms
            (e.g., NR mergers, SNe) but some maintenance required, details TBD
            for SNe
        \item Swig-wrapped {\tt LAL}: high-level routines for waveform
            generation \& injection - useful for development \& plotting
        \item Wiki to collate info \& examples coming soon!
    \end{itemize}
\end{frame}


\end{document}
